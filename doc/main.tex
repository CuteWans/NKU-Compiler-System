% -*- coding: utf-8 -*-
%-------------------------designed by zcf--------------
\documentclass[UTF8,a4paper,10pt]{ctexart}
\usepackage[left=3.17cm, right=3.17cm, top=2.74cm, bottom=2.74cm]{geometry}
\usepackage{amsmath}
\usepackage{graphicx,subfig}
\usepackage{float}
\usepackage{cite}
\usepackage{caption}
\usepackage{enumerate}
\usepackage{booktabs} %表格
\usepackage{multirow}
\usepackage{minted}
\usepackage{svg}
\newcommand{\tabincell}[2]{\begin{tabular}{@{}#1@{}}#2\end{tabular}}  %表格强制换行
%-------------------------字体设置--------------
% \usepackage{times} 
\usepackage{ctex}
\setCJKmainfont[ItalicFont=Noto Sans CJK SC Bold, BoldFont=Noto Serif CJK SC Black]{Noto Serif CJK SC}
\newcommand{\yihao}{\fontsize{26pt}{36pt}\selectfont}           % 一号, 1.4 倍行距
\newcommand{\erhao}{\fontsize{22pt}{28pt}\selectfont}          % 二号, 1.25倍行距
\newcommand{\xiaoer}{\fontsize{18pt}{18pt}\selectfont}          % 小二, 单倍行距
\newcommand{\sanhao}{\fontsize{16pt}{24pt}\selectfont}  %三号字
\newcommand{\xiaosan}{\fontsize{15pt}{22pt}\selectfont}        % 小三, 1.5倍行距
\newcommand{\sihao}{\fontsize{14pt}{21pt}\selectfont}            % 四号, 1.5 倍行距
\newcommand{\banxiaosi}{\fontsize{13pt}{19.5pt}\selectfont}    % 半小四, 1.5倍行距
\newcommand{\xiaosi}{\fontsize{12pt}{18pt}\selectfont}            % 小四, 1.5倍行距
\newcommand{\dawuhao}{\fontsize{11pt}{11pt}\selectfont}       % 大五号, 单倍行距
\newcommand{\wuhao}{\fontsize{10.5pt}{15.75pt}\selectfont}    % 五号, 单倍行距
%-------------------------章节名----------------
\usepackage{ctexcap} 
\CTEXsetup[name={,、},number={ \chinese{section}}]{section}
\CTEXsetup[name={(,)},number={\chinese{subsection}}]{subsection}
\CTEXsetup[name={,.},number={\arabic{subsubsection}}]{subsubsection}
%-------------------------页眉页脚--------------
\usepackage{fancyhdr}
\pagestyle{fancy}
\lhead{\kaishu \leftmark}
% \chead{}
\rhead{\kaishu 预习报告}%加粗\bfseries 
\lfoot{}
\cfoot{\thepage}
\rfoot{}
\renewcommand{\headrulewidth}{0.1pt}  
\renewcommand{\footrulewidth}{0pt}%去掉横线
\newcommand{\HRule}{\rule{\linewidth}{0.5mm}}%标题横线
\newcommand{\HRulegrossa}{\rule{\linewidth}{1.2mm}}
%-----------------------伪代码------------------
\usepackage{algorithm}  
\usepackage{algorithmicx}  
\usepackage{algpseudocode}  
\floatname{algorithm}{Algorithm}  
\renewcommand{\algorithmicrequire}{\textbf{Input:}}  
\renewcommand{\algorithmicensure}{\textbf{Output:}} 
\usepackage{lipsum}  
\makeatletter
\newenvironment{breakablealgorithm}
  {% \begin{breakablealgorithm}
  \begin{center}
     \refstepcounter{algorithm}% New algorithm
     \hrule height.8pt depth0pt \kern2pt% \@fs@pre for \@fs@ruled
     \renewcommand{\caption}[2][\relax]{% Make a new \caption
      {\raggedright\textbf{\ALG@name~\thealgorithm} ##2\par}%
      \ifx\relax##1\relax % #1 is \relax
         \addcontentsline{loa}{algorithm}{\protect\numberline{\thealgorithm}##2}%
      \else % #1 is not \relax
         \addcontentsline{loa}{algorithm}{\protect\numberline{\thealgorithm}##1}%
      \fi
      \kern2pt\hrule\kern2pt
     }
  }{% \end{breakablealgorithm}
     \kern2pt\hrule\relax% \@fs@post for \@fs@ruled
  \end{center}
  }
\makeatother
%------------------------代码-------------------
\usepackage{xcolor} 
\usepackage{listings} 
\lstset{ 
breaklines,%自动换行
basicstyle=\small,
escapeinside=``,
keywordstyle=\color{ blue!70} \bfseries,
commentstyle=\color{red!50!green!50!blue!50},% 
stringstyle=\ttfamily,% 
extendedchars=false,% 
linewidth=\textwidth,% 
numbers=left,% 
numberstyle=\tiny \color{blue!50},% 
frame=trbl% 
rulesepcolor= \color{ red!20!green!20!blue!20} 
}
%------------超链接----------
\usepackage[colorlinks,linkcolor=black,anchorcolor=blue]{hyperref}
%------------------------TODO-------------------
\usepackage{enumitem,amssymb}
\newlist{todolist}{itemize}{2}
\setlist[todolist]{label=$\square$}
% for check symbol 
\usepackage{pifont}
\newcommand{\cmark}{\ding{51}}%
\newcommand{\xmark}{\ding{55}}%
\newcommand{\done}{\rlap{$\square$}{\raisebox{2pt}{\large\hspace{1pt}\cmark}}\hspace{-2.5pt}}
\newcommand{\wontfix}{\rlap{$\square$}{\large\hspace{1pt}\xmark}}
%------------------------水印-------------------
\usepackage{tikz}
\usepackage{xcolor}
\usepackage{eso-pic}

\newcommand{\watermark}[3]{\AddToShipoutPictureBG{
\parbox[b][\paperheight]{\paperwidth}{
\vfill%
\centering%
\tikz[remember picture, overlay]%
  \node [rotate = #1, scale = #2] at (current page.center)%
    {\textcolor{gray!80!cyan!30!magenta!30}{#3}};
\vfill}}}

%———————————————————————————————————————————正文———————————————————————————————————————————————
%----------------------------------------------
\begin{document}
\begin{titlepage}
  \begin{center}
    \includegraphics[width=0.8\textwidth]{figure/NKU.png}\\[1cm]
    \textsc{\Huge \kaishu{\textbf{南\ \ \ \ \ \ 开\ \ \ \ \ \ 大\ \ \ \ \ \ 学}} }\\[0.9cm]
    \textsc{\huge \kaishu{\textbf{计\ \ 算\ \ 机\ \ 学\ \ 院}}}\\[0.5cm]
    % \textsc{\Large \textbf{预习报告}}\\[0.8cm]
    \HRule \\[0.9cm]
    { \LARGE \bfseries 预习报告}\\[0.4cm]
    \HRule \\[2.0cm]
    \centering
    \textsc{\LARGE 丁屹\kaishu{\ \ \ \ }}\\[0.5cm]
    \textsc{\LARGE \kaishu{年级\ :\ 2020级}}\\[0.5cm]
    \textsc{\LARGE \kaishu{专业\ :\ 计算机科学与技术}}\\[0.5cm]
    % \textsc{\LARGE \kaishu{指导教师\ :\ XX}}\\[0.5cm]
    \vfill
    {\Large \today}
  \end{center}
\end{titlepage}
%-------------摘------要--------------
% \newpage
% \thispagestyle{empty}
% \renewcommand{\abstractname}{\kaishu \sihao \textbf{摘要}}
% \begin{abstract}

%   \noindent  %顶格
%   \textbf{\\\ 关键字:XX、YY、ZZ}\textbf{} \\\ \\\
% \end{abstract}
%----------------------------------------------------------------
\tableofcontents
%----------------------------------------------------------------
\newpage
\watermark{60}{10}{NKU}
\setcounter{page}{1}

\section{作业1}

对于以下两种代码,选择 LOOP \#2 可以得到更快的执行速度。

\begin{minted}[linenos,frame=lines]{c}
  /* LOOP #1 */
  int i;
  for (i = 0; i < N; ++i) {
    a[i] *= 2000;
    a[i] /= 10000;
  }
\end{minted}

\begin{minted}[linenos,frame=lines]{c}
  /* LOOP #2 */
  int* b = a;
  int  i;
  for (i = 0; i < N; ++i) {
    *b *= 2000;
    *b /= 10000;
    ++b;
  }
\end{minted}

在x86\_64 CPU,Arch Linux系统使用GCC 12.2.0编译器测试,其中N取值1000000000,取4次用时的平均值,计时器使用clock\_gettime(CLOCK\_MONOTONIC, \&ts),代码位于src/loop1.c与src/loop2.c。

如果不打开-O优化,可以得到 LOOP \#1 平均用时 5049361472 ns,得到 LOOP \#2 平均用时 2849261662 ns。LOOP \#2速度更快。

\begin{minted}[linenos,frame=lines]{ASM}
  a:
    .zero   4000000000
  main:
    push    rbp
    mov     rbp, rsp
    mov     DWORD PTR [rbp-4], 0
    jmp     .L2
  .L3:
    mov     eax, DWORD PTR [rbp-4]
    cdqe
    mov     eax, DWORD PTR a[0+rax*4]
    imul    edx, eax, 2000
    mov     eax, DWORD PTR [rbp-4]
    cdqe
    mov     DWORD PTR a[0+rax*4], edx
    mov     eax, DWORD PTR [rbp-4]
    cdqe
    mov     eax, DWORD PTR a[0+rax*4]
    movsx   rdx, eax
    imul    rdx, rdx, 1759218605
    shr     rdx, 32
    sar     edx, 12
    sar     eax, 31
    sub     edx, eax
    mov     eax, DWORD PTR [rbp-4]
    cdqe
    mov     DWORD PTR a[0+rax*4], edx
    add     DWORD PTR [rbp-4], 1
  .L2:
    cmp     DWORD PTR [rbp-4], 999999999
    jle     .L3
    mov     eax, 0
    pop     rbp
    ret
\end{minted}

\begin{minted}[linenos,frame=lines]{ASM}
  a:
    .zero   4000000000
  main:
    push    rbp
    mov     rbp, rsp
    mov     QWORD PTR [rbp-8], OFFSET FLAT:a
    mov     DWORD PTR [rbp-12], 0
    jmp     .L2
  .L3:
    mov     rax, QWORD PTR [rbp-8]
    mov     eax, DWORD PTR [rax]
    imul    edx, eax, 2000
    mov     rax, QWORD PTR [rbp-8]
    mov     DWORD PTR [rax], edx
    mov     rax, QWORD PTR [rbp-8]
    mov     eax, DWORD PTR [rax]
    movsx   rdx, eax
    imul    rdx, rdx, 1759218605
    shr     rdx, 32
    sar     edx, 12
    sar     eax, 31
    sub     edx, eax
    mov     rax, QWORD PTR [rbp-8]
    mov     DWORD PTR [rax], edx
    add     QWORD PTR [rbp-8], 4
    add     DWORD PTR [rbp-12], 1
  .L2:
    cmp     DWORD PTR [rbp-12], 999999999
    jle     .L3
    mov     eax, 0
    pop     rbp
    ret
\end{minted}

比较二者的汇编代码可以发现主要的区别在于:LOOP \#1在每次访问数组时都会计算0+rax*4,做下标转换;而LOOP \#2中每次只对指针[rbp-8]+4,计算量更小。

如果开启-O2优化,可以得到 LOOP \#1 平均用时 1652242168 ns,得到 LOOP \#2 平均用时 1664580829 ns。可以认为LOOP \#1与LOOP \#2没有性能差距。此时二者汇编代码没有区别,优化为使用SIMD超标量技术加速。

\section{作业2}
分词
\begin{itemize}
  \item Model (field)
  \item = (boolean operator)
  \item " (string begin)
  \item Civic (string content)
  \item " (string end)
  \item AND (boolean operator)
  \item Year (field)
  \item = (boolean operator)
  \item " (string begin)
  \item 2001 (string content)
  \item " (string end)
\end{itemize}

\begin{figure}[H]
  \centering
  \includesvg[width=\textwidth]{figure/syntax.svg}
  \caption{语法分析树}
\end{figure}

\section{作业3}
使用splint可以得到如下输出
\begin{minted}[linenos,frame=lines]{text}
Splint 3.1.2a --- May 25 2020

src/static-check.c: (in function f)
src/static-check.c:13:10: Stack-allocated storage &loc reachable from return
                             value: &loc
  A stack reference is pointed to by an external reference when the function
  returns. The stack-allocated storage is destroyed after the call, leaving a
  dangling reference. (Use -stackref to inhibit warning)
src/static-check.c:13:10: Immediate address &loc returned as implicitly only:
                             &loc
  An immediate address (result of & operator) is transferred inconsistently.
  (Use -immediatetrans to inhibit warning)
src/static-check.c:13:15: Stack-allocated storage *x reachable from parameter x
   src/static-check.c:12:3: Storage *x becomes stack-allocated storage
src/static-check.c:13:15: Function returns with global glob referencing
                             released storage
  A global variable does not satisfy its annotations when control is
  transferred. (Use -globstate to inhibit warning)
   src/static-check.c:13:10: Storage glob released
src/static-check.c: (in function h)
src/static-check.c:18:7: Comparison of unsigned value involving zero: i >= 0
  An unsigned value is used in a comparison with zero in a way that is either a
  bug or confusing. (Use -unsignedcompare to inhibit warning)
src/static-check.c:18:7: Variable i used before definition
  An rvalue is used that may not be initialized to a value on some execution
  path. (Use -usedef to inhibit warning)
src/static-check.c:7:6: Variable exported but not used outside static-check:
                           glob
  A declaration is exported, but not used outside this module. Declaration can
  use static qualifier. (Use -exportlocal to inhibit warning)

Finished checking --- 7 code warnings
\end{minted}
\begin{itemize}
  \item 在函数f内,返回了指向局部变量loc的指针,可能会导致释放后使用
  \item 在函数f内,传入的参数x指向的指针*x被写入了局部变量loc的地址,函数返回后访问**x会导致释放后使用
  \item 在函数h内,未初始化变量i就访问其值
  \item 在函数h内,变量i类型是无符号整数,只能走$i \geq 0$分支,另一分支无意义
  \item 在函数firstChar1内,没有对传入指针做检查,可能会导致解引用NULL(此处splint不认为有问题)
\end{itemize}

\section{作业4}
\begin{enumerate}
  \item int a;是变量声明
  \item int a;中a是标识符列表
  \item 若a是标识符列表,则a, b也是标识符列表
\end{enumerate}

\end{document}
